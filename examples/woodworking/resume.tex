\documentclass[10pt]{article}
\usepackage[margin=0.75in]{geometry}
\usepackage{enumitem}
\usepackage{titlesec}
\usepackage{hyperref}
\usepackage{xcolor}

% Customize hyperlink appearance
\definecolor{darkblue}{RGB}{0,0,120}
\hypersetup{
    colorlinks=true,
    linkcolor=darkblue,
    filecolor=darkblue,
    urlcolor=darkblue,
}

% No paragraph indentation
\setlength{\parindent}{0pt}
% \setlength{\parskip}{0.5em} % Optional: adds space between paragraphs

% Section formatting
\titleformat{\section}
    {\normalsize\bfseries\uppercase}
    {}{0em}{}[\titlerule]
\titlespacing*{\section}
    {0pt}{1.5ex plus 1ex minus .2ex}{1ex plus .2ex}

% Custom commands for better spacing
\newcommand{\spaced}[1]{\vspace{0.3em}#1\vspace{0.3em}}

\begin{document}

% Header
\begin{center}
    {\Large\bfseries{Geppetto}}\\[0.2em]
    {\normalsize
        \href{mailto:geppetto@woodmail.it}{geppetto@woodmail.it} ~|~
        \href{tel:+39 123 456 7890}{+39 123 456 7890} ~|~
        \href{https://github.com/geppetto-works}{GitHub}
    }
\end{center}

\vspace{0.5em}

% Summary

\section*{Summary}

Master woodcarver with a lifelong dedication to toy design and mechanical storytelling.
Combines traditional techniques with creative engineering to build expressive works.
Seeking to contribute craftsmanship to meaningful creative projects.




% Experience
\section*{Experience}

\textbf{Geppetto's Workshop}, \hfill Villaggio Incantato, Italy \\
\textit{Master Woodcarver and Toymaker}, \hfill January 1850 -- Present
\begin{itemize}[leftmargin=*, noitemsep, topsep=0.1em]

    \item Designed and handcrafted over 300 wooden toys including animals, puppets, and automata.

    \item Created Pinocchio, an autonomous puppet capable of emotional response.

    \item Offered sliding-scale toy repairs and creations for families in the village.

\end{itemize}
\vspace{0.3em}

\textbf{Puppet Theater of Tuscany}, \hfill Tuscany, Italy \\
\textit{Set Designer and Puppet Articulator}, \hfill June 1848 -- December 1849
\begin{itemize}[leftmargin=*, noitemsep, topsep=0.1em]

    \item Designed stage mechanisms and built expressive puppets for shows.

    \item Enhanced puppet articulation through mechanical design.

\end{itemize}
\vspace{0.3em}


% Projects
\section*{Projects}

\textbf{\href{https://en.wikipedia.org/wiki/Pinocchio}{Pinocchio}}, \hfill Ongoing \\
A fully articulated wooden puppet designed for emotional expressiveness and narrative-constrained learning.
 \\
\vspace{0.3em}

\textbf{\href{}{Whistle-toy Orchestra}}, \hfill March 1885 \\
A set of carved whistle-operated instruments teaching children musical structure.
 \\
\vspace{0.3em}


% Education
\section*{Education}

\textbf{Guild of Master Carvers, } \hfill Florence, Italy \\
Journeyman Woodworker Certification, \hfill 1842
\vspace{0.3em}

\textbf{Independent Study, } \hfill Venice, Italy \\
Apprenticeship in Automata Construction and Puppet Mechanics, \hfill 1845
\vspace{0.3em}


% Honors and Awards
\section*{Honors and Awards}

\textbf{Best Toymaker in Tuscany,} \hfill December 1883 \\
Awarded by the Tuscany Chamber of Artisans for craftsmanship and imaginative design.
\vspace{0.3em}

\textbf{Patron of the Children's Guild,} \hfill June 1886 \\
Recognized for mentorship and support of creative expression in youth.
\vspace{0.3em}


% Publications
\section*{Publications}
\begin{itemize}[leftmargin=*, noitemsep, topsep=0.1em]

    \item \textit{\href{}{Toymaking for the Soul: A Handbook for Village Craft Educators}}, with Silvino Boccadutti. Community Education Manual.

    \item \textit{\href{}{On the Practical Construction of Puppet Joints}}. Instructional Manual.

\end{itemize}

\end{document}